%%%%%%%% ICML 2019 EXAMPLE LATEX SUBMISSION FILE %%%%%%%%%%%%%%%%%

\documentclass{article}

\usepackage{kotex}
% Recommended, but optional, packages for figures and better typesetting:
\usepackage{microtype}
\usepackage{graphicx}
\usepackage{subfigure}
\usepackage{booktabs} % for professional tables

% hyperref makes hyperlinks in the resulting PDF.
% If your build breaks (sometimes temporarily if a hyperlink spans a page)
% please comment out the following usepackage line and replace
% \usepackage{icml2019} with \usepackage[nohyperref]{icml2019} above.
\usepackage{hyperref}

% Attempt to make hyperref and algorithmic work together better:
\newcommand{\theHalgorithm}{\arabic{algorithm}}

% Use the following line for the initial blind version submitted for review:
%\usepackage{icml2019}

% If accepted, instead use the following line for the camera-ready submission:
\usepackage[accepted]{icml2019}

% The \icmltitle you define below is probably too long as a header.
% Therefore, a short form for the running title is supplied here:
\icmltitlerunning{COSE474-2019F: Final Project Proposal}

\begin{document}

\twocolumn[
\icmltitle{COSE474-2019F: Final Project Proposal \\
           Improving Detecting DeepFake images with 어쩌고저쩌고}

% It is OKAY to include author information, even for blind
% submissions: the style file will automatically remove it for you
% unless you've provided the [accepted] option to the icml2019
% package.

% List of affiliations: The first argument should be a (short)
% identifier you will use later to specify author affiliations
% Academic affiliations should list Department, University, City, Region, Country
% Industry affiliations should list Company, City, Region, Country

% You can specify symbols, otherwise they are numbered in order.
% Ideally, you should not use this facility. Affiliations will be numbered
% in order of appearance and this is the preferred way.
\icmlsetsymbol{equal}{*}

\begin{icmlauthorlist}
\icmlauthor{제한재}{}
\icmlauthor{허환}{}
\icmlauthor{홍주연}{}
\icmlauthor{도재형}{}
\icmlauthor{채병주}{}
\end{icmlauthorlist}

%\icmlaffiliation{ku}{Department of Computer Science \& Engineering, Korea University, Seoul, Korea}


%\icmlcorrespondingauthor{the}{myemail@korea.ac.kr}
%\icmlcorrespondingauthor{Eee Pppp}{ep@eden.co.uk}

% You may provide any keywords that you
% find helpful for describing your paper; these are used to populate
% the "keywords" metadata in the PDF but will not be shown in the document
\icmlkeywords{Machine Learning, ICML}

\vskip 0.3in
]

% this must go after the closing bracket ] following \twocolumn[ ...

% This command actually creates the footnote in the first column
% listing the affiliations and the copyright notice.
% The command takes one argument, which is text to display at the start of the footnote.
% The \icmlEqualContribution command is standard text for equal contribution.
% Remove it (just {}) if you do not need this facility.

%\printAffiliationsAndNotice{}  % leave blank if no need to mention equal contribution
%\printAffiliationsAndNotice{\icmlEqualContribution} % otherwise use the standard text.

\begin{abstract}
제곧내
\end{abstract}

\section{Intruduction}

\subsection{Deepfakes}

Deepfake는 어떤 것을 말한다. (대충 인터넷에 많고 사생활 침해도 우려된다는 내용). Deepfake를 생성하는 대표적인 방법은 generative adversarial networ (GAN)을 이용하는 방법이다.

따라서 이번 프로젝트에서 Deep

\section{Datasets}

Google에서 공개한 Deepfake Detection Dataset \cite{DDD_GoogleJigSaw2019} 을 사용할 예정이다. 해당 dataset은 FaceForensics++ \cite{roessler2019faceforensicspp}의 GitHub 
페이지에서 내려받을 수 있다.

\section{Goals}

우리의 목표는 새로이 공개된 데이터셋에 대해 강하게 학습할 수 있는 모델을 생성하고 두 가지 모델 지표 중 하나를 달성하는 것이다. 첫 번째는 Sensitivity가 높은 모델을 개발하여 현실 세계에서 이미지의 위험을 성공적으로 경고하는 모델을 생성하는 것이고, 더 나아가 높은 F-Score를 달성한 쌍방향으로 신뢰가능한 모델을 생성하는 것이다.

\section{Brief Schedule}

\section{Roles}

우리 팀은 총 5명으로 이루어져 있다. 모든 팀원이 모델 기획 및 구현에 참여한다. 추가로, 기타 여러 가지 업무를 다음과 같이 분담하여 진행하기로 하였다.
\begin{itemize}
	\item 제한재 -- Data preprocessing
	\item 도재형, 홍주연 -- SOTA model \& dataset research
	\item 채병주, 허환 -- Model fine-tuning
\end{itemize}

\section{Comparison with SOTA}

Yuezen et~al.은 DeepFake 로 생성된 비디오에서는 눈을 감고 있는 이미지에 대해 모델의 학습이 부족하여 정상적으로 눈 깜빡임을 판별하는 CNN/RNN 모델을 제안한 바 있다 \yrcite{yuezeun2018eyeblink}. David et~al.의 연구에서는 RNN과 CNN에서 긍정적인 결과를 얻었지만 전체적인 접근 방식에는 약점이 있는 것을 보였다 \yrcite{DavidDetection}. 더 나아가 생성된 데이터를 바탕으로 학습하는것이 효율적이지 못하고 실제 이미지와 가짜 이미지를 동시에 학습이 필요하다고 제시하고 있다. 개선된 데이터셋을 바탕으로 더 기존의 학습 모델에서 더 나은 퍼포먼스를 보이는 모델을 생성하고자 한다.


% In the unusual situation where you want a paper to appear in the
% references without citing it in the main text, use \nocite


\bibliography{proposal}
\bibliographystyle{icml2019}


\end{document}


% This document was modified from the file originally made available by
% Pat Langley and Andrea Danyluk for ICML-2K. This version was created
% by Iain Murray in 2018, and modified by Alexandre Bouchard in
% 2019. Previous contributors include Dan Roy, Lise Getoor and Tobias
% Scheffer, which was slightly modified from the 2010 version by
% Thorsten Joachims & Johannes Fuernkranz, slightly modified from the
% 2009 version by Kiri Wagstaff and Sam Roweis's 2008 version, which is
% slightly modified from Prasad Tadepalli's 2007 version which is a
% lightly changed version of the previous year's version by Andrew
% Moore, which was in turn edited from those of Kristian Kersting and
% Codrina Lauth. Alex Smola contributed to the algorithmic style files.
